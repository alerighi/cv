\documentclass[12pt,a4paper,sans]{moderncv}
\usepackage[english]{babel}
\usepackage[utf8]{inputenc}
\usepackage[margin=1.5cm]{geometry}

\moderncvstyle{classic}
\definecolor{color0}{rgb}{0,0,0}% black
\definecolor{color1}{rgb}{0.35,0.65,0.40}% green
\definecolor{color2}{rgb}{0.45,0.45,0.45}% dark grey

\firstname{Alessandro}
\lastname{Righi}
\title{Curriculum Vitae}
\address{Via Alessandro Volta 9}{37029 San Pietro in Cariano (VR)}{Italy}
\phone[mobile]{+39~349~843~7935}
\email{alessandro.righi@outlook.it}
\homepage{alerighi.it}
\social[linkedin]{alerighi}
\social[github]{alerighi}

\begin{document}
\makecvtitle{}

\section{Education}
\cventry{2018--2022}{Master degree in Computer Science and Engineering}{University of Verona}{}{\textit{109/110}}{In 
    my master degree program I choose to specialize in the curriculum of cybersecurity, and get into more details of 
    the theory behind Computer Science.}
\cventry{2015--2018}{Bachelor's degree in Computer Science}{University of Verona}{}{\textit{110/110 with honors}}{In 
    my bachelor degree in computer science I've got into more knowledge of all the CS aspects and their theoretical 
    foundations, from operating systems, to databases, to networks and much more.}
\cventry{2010--2015}{Diploma of ordinary scientific lyceum}{ISIS L.Calabrese P.Levi}{San Pietro in Cariano (VR)}{\textit{80/100}}{
    During high-school I didn't touch a computer, but instead studied latin and philosophy. During this period I 
    discovered and learned programming by myself while avoiding studying latin.}

\section{Master thesis}
\cvitem{title}{\emph{Automation of acceptance tests for embedded IoT devices integrated with the cloud}}
\cvitem{supervisor}{Mariano Ceccato}
\cvitem{description}{IoT devices are ubiquitous these days, and serve critical functions in our lives. 
    Manufacturers have the possibility to perform \textit{Over The Air (OTA)} updates to add features and 
    fix bugs. However each software release may add new bugs, that shall be avoided at all costs. 
    Manual testing of each firmware update is a time consuming and error-prone operation. In my thesis I 
    present a framework that I've created to automate the execution of such tests. The system doesn't emulate 
    the firmware but rather executes it on the production microcontroller, that is treated as a black-box,
    ensuring that also bugs caused by the integration with the hardware are caught. The system interacts
    with the device either physically and with the cloud, and verifies that the IoT device performs the correct
    functionalities, as specified in simple to write test cases.}

\section{Experience}
% \cventry{year--year}{Job title}{Employer}{City}{}{General description no longer than 1--2 lines}
\cventry{2020--today}{Applied Research Director}{IOTINGA s.r.l.}{Verona (IT)}{}{I started workin in IOTINGA in 2020
    as a software developer, and now I hold the position of \textit{Applied Research Director}. My jobs are exploring
    and testing new technologies and designing the system architecture that better solves our customer problems, as
    well as writing code (that I will never renounce doing!). We mainly operate in the field of IoT and IIoT/industrial
    automation, and solve our customer problems with cutting edge full stack solutions, that range from the firmware
    of the embedded device, to the backend/cloud, to finally consumer iOS/Android apps and web applications.}
\cventry{2021--today}{Instructor for the CyberChallenge.IT program}{University of Verona}{}{}{I'm one of the main 
    organizers for the Cybersecurity training program CyberChallenge.IT at my former university. Here I train students
    on cybersecurity, in particular applied to web application security.}
\cventry{2018--2022}{Tutor for the Italian Olympics in Informatics team}{AICA}{Remote}{}{I partecipated as a tutor 
    for the Italian Olympics in Informatics. Here I assisted at the training of the national team, and I helped 
    develop software and tools used in the program.}
\cventry{2018--2020}{Organizer for the course Programming Challenges}{University of Verona}{}{}{This is a self-managed 
    course that we and other collegues organized for the student to get into competitive programming and prepare for 
    contests, such as the SWERC or Google Hashcode competitions.}
\cventry{2019--2020}{Tutor for the courses Programming I, Bioinformatics}{University of Verona}{}{}{I assisted in the
    lab first-year students in bioinformatics learning programming (in C) for the first time.}
\cventry{2019--2020}{Tutor for the courses Programming II, Informatics}{University of Verona}{}{}{I assisted in the lab
    second-year CS students learning Java and OOP programming.}
\cventry{2018--2019}{Tutor for the course Programming, Applied Mathematics}{University of Verona}{}{}{I assisted math
    students getting into programming (in Python) for the first time.}
\cventry{occasional}{Farmer}{}{}{}{I used to help my granddad and my dad with the work in the fields, in particular 
    regarding grape and cherry harvesting.}

\section{Archivements}
\cventry{2022}{Third place at the Würth Phoenix CTF}{Würth Phoenix}{}{}{With my team we qualified 3rd at the CTF organized by Würth Phoenix,
    solving 5 out of 6 web security challenges, and winning a pize of 500 euros.}
\cventry{2020}{Third place at the CyberChallenge.IT national challenge}{University of Verona}{}{}{With my team we qualified 
    third place at the finals for the CyberChallenge.IT program, consisting in an A/D CTF competition to which participated 
    more than 20 other universities.}
\cventry{2020}{First place at the local CyberChallenge.IT CTF}{University of Verona}{}{}{I qualified first at the local 
    CyberChallenge.IT CTF finals in my university.}
\cventry{2020}{35st place at the ICPC-ACM SWERC 2019-2020}{Télécom ParisTech}{Paris}{}{With my team we qualified 35st 
    to the Southwestern Europe Regional Contest (SWERC) competitive programming competition, to which participated more 
    than 100 teams.}
\cventry{2019}{21st place at the Google HashCode word finals}{Google European Headquarter}{Dublin}{}{With my team we 
    were one of the 40 teams selected to participate in the Google Hashcode finals at the Google Headquarters in Dublin. 
    Here we arrived 21st.}
\cventry{2018}{22nd place at the ICPC-ACM SWERC 2018-2019}{Télécom ParisTech}{Paris}{}{With my team we qualified 22nd 
    to the Southwestern Europe Regional Contest (SWERC) competitive programming competition, to which participated more than 100 teams.}

\section{Languages}
\cvitemwithcomment{Italian}{Native speaker}{}
\cvitemwithcomment{English}{Level B1}{}

\section{Technical skills}
\cvitem{Backend}{I developed multiple backends using a multitude of technologies, I've a good knowledge of SQL and NoSQL databases}
\cvitem{Linux System administration}{I've 10 years experience using GNU/Linux systems, where I experimented with a lot of distributions. I've experience administering a server and relative infrastructure}
\cvitem{App}{I've developed multiple iOS/Android applications, mainly using React Native, but also writing some native code}
\cvitem{Frontend}{I know the latest technologies and standards for writing web applications, and the main frameworks in particular React/Next.JS}
\cvitem{Cybersecurity}{I'm a cybersecurity enthusiast, I've participated in a lot of CTF competitions, and I've epxerience especially for what concerns web application security and networking protocols}
\cvitem{Cloud}{I've used AWS services a lot, and I've experience in designing a serverless architecture and managing IoT devices with IoT Core}
\cvitem{Embedded}{I've experience in developing firmware for 32-bit embedded microcontrollers, such as ESP-32 and similar, that interfaces with the cloud torugh MQTT}
\cvitem{Electronics}{I've good knowledge of digital electronics, I'm able to design simple PCBs with KiCad, and I've experience in repairing and troubleshooting}

\section{Interests}
\cvitem{Free Software}{I'm a huge supported of the free software movement and use as much as I can open software and platforms. I use Linux since I was 15, and I've published some project on my GitHub.}
\cvitem{Retrocomputing}{I like to collect and repair piece of old technology, and like to study how old computer systems works. I dream owning an IBM AS/400!}
\cvitem{DIY}{I like learning to do stuff myself, I try to repair every kind of object that breaks, do electric work, hack things to make them better, from a child I always had a screwdriver in my hands.}
\cvitem{Engineering}{I'm fascinated by engineering and want to know as much as possible how every complex thing works, from bridges, to dams, electricity transmission lines, nuclear power plants, etc.}
\cvitem{Cycling}{After a day of hard work in front of a screen I like to ride my bike in the hills around my city.}
\cvitem{Hiking}{I like to hike in the mountains, observe nature and relax in peaceful places.}

\vspace*{\fill}

You can get the latest version of this document here: \url{https://alerighi.it/cv/cv-alessandro-righi.pdf}. \\
Take also a loot at my website \url{https://alerighi.it} and my GitHub page \url{https://github.com/alerighi}.
\end{document}
