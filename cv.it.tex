\documentclass[11pt,a4paper,sans]{moderncv}
\usepackage[italian]{babel}
\usepackage[utf8]{inputenc}
\usepackage[margin=0.55in]{geometry}

\moderncvstyle{classic}
\definecolor{color0}{rgb}{0,0,0}% black
\definecolor{color1}{rgb}{0.35,0.65,0.40}% green
\definecolor{color2}{rgb}{0.45,0.45,0.45}% dark grey

\firstname{Alessandro}
\lastname{Righi}
\title{Curriculum vitae}
\address{Verona, Veneto}{Italia}
\homepage{alerighi.it}
\social[linkedin]{alerighi}
\social[github]{alerighi}

\begin{document}
\makecvtitle{}

Ho sempre avuto una forte passione per l'informatica e l'elettronica. Attualmente lavoro in IOTINGA, una giovane azienda
specializzata in dispositivi IoT ed industria 4.0. Ho ottenuto una laurea magistrale in ingegneria e scienze informatiche con
specializzazione in sicurezza informatica presso l'Università degli studi di Verona. Nel tempo libero sono istruttore per
il progetto CyberChallenge.IT nella sede UniVR. In passato ho fatto il tutor per le Olimpiadi dell'Informatica.

\section{Istruzione}
\cventry{2018--2022}{Laurea magistrale in Ingegneria e scienze informatiche, indirizzo Sicurezza}{Università degli studi di Verona}{}{\textit{109/110}}{}
\cventry{2015--2018}{Laurea in Informatica}{Università degli studi di Verona}{}{\textit{110/110 con Lode}}{}
\cventry{2010--2015}{Diploma di liceo scientifico}{ISIS L.Calabrese P.Levi}{San Pietro in Cariano (VR)}{\textit{80/100}}{}

\section{Esperienze}
\cventry{2020--oggi}{Software engineer}{IOTINGA s.r.l.}{San Martino Buon Albergo (VR)}{}{}
\cventry{2021--oggi}{Istruttore per il progetto CyberChallenge.IT}{Università degli studi di Verona}{}{}{}
\cventry{2018--2022}{Tutor e sviluppatore software presso il team Olimpiadi italiane di Informatica}{}{}{}{}
\cventry{2018--2020}{Tutor del corso ``Sfide di programmazione''}{Università degli studi di Verona}{}{}{}
\cventry{2019--2020}{Tutor del corso ``Programmazione I'', Bioinformatica}{Università degli studi di Verona}{}{}{}
\cventry{2019--2020}{Tutor del corso ``Programmazione II'', Informatica}{Università degli studi di Verona}{}{}{}
\cventry{2018--2019}{Tutor del corso ``Programmazione'', Matematica Applciata}{Università degli studi di Verona}{}{}{}

\section{Riconoscimenti}
\cventry{2022}{CES Best of Innovation con il radiatore Polygon}{IOTINGA s.r.l. ed IRSAP s.p.a.}{Las Vegas}{}{}
\cventry{2020}{Terzo classificato alla finale nazionale di CyberChallenge.IT}{Università degli studi di Verona}{}{}{}
\cventry{2020}{Primo posto alla finale locale di CyberChallenge.IT}{Università degli studi di Verona}{}{}{}
\cventry{2020}{35-esimi classificati, ICPC-ACM SWERC 2019/2020}{Télécom ParisTech}{Parigi}{}{}
\cventry{2019}{21-esimi classificati, finali mondiali Google HashCode}{Google headquarter}{Dublino}{}{}
\cventry{2018}{22-esimi classificati, ICPC-ACM SWERC 2018/2019}{Télécom ParisTech}{Parigi}{}{}

\section{Lingue}
\cvitemwithcomment{Italiano}{Lingua madre}{}
\cvitemwithcomment{Inglese}{Livello B1}{}

\section{Competenze tecniche}
\cvitem{Embedded}{Sviluppo firmware per microcontrollori 32-bit connessi, MQTT, BLE, Modbus, Espressif, FreeRTOS}
\cvitem{App}{Realizzazione di app iOS/Android con React Native}
\cvitem{Backend}{Realizzazione di backend REST, Node.JS, Python, Django, PostgreSQL}
\cvitem{Cloud}{Ottima conoscenza dei servizi di AWS, serverless, DynamoDB, EC2, ECS, Lambda, IoT Core, IAM, Cloudformation}
\cvitem{Elettronica}{Progettazione di circuiti elettronici digitali, realizzazione PCB con KiCad, diagnosi e riparazione di guasti}
\cvitem{Amministrazione di sistema}{Amministrazione di sistemi e server GNU/Linux, Docker, Proxmox VE, sistemi CI/CD, Ansible}
\cvitem{Cybersecurity}{Penetration testing, analisi di malware, reverse engineering, CTF}
\cvitem{Frontend}{Realizzazione applicazioni web React, CSS, TypeScript, PHP}

\vspace*{\fill}

Puoi ottenere una versione aggiornata di questo documento qui: \url{https://alerighi.it/cv-alessandro-righi.it.pdf}. \\
Fai anche un salto sul mio sito \url{https://alerighi.it} ed il mio profilo GitHub \url{https://github.com/alerighi}.

\end{document}

