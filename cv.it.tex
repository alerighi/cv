\documentclass[11pt,a4paper,sans]{moderncv}
\usepackage[italian]{babel}
\usepackage[utf8]{inputenc}
\usepackage[scale=0.75]{geometry}

\moderncvstyle{classic} 
\moderncvcolor{blue} 

\name{Alessandro}{Righi}
\title{Curriculum vitae}
\address{Via A.Volta 9}{San Pietro in Cariano (VR)}{37029}
\phone[mobile]{+39~349~843~7935}
\email{alessandro.righi@outlook.it}
\homepage{alerighi.github.io}
\social[linkedin]{alerighi}
\social[github]{alerighi}

\begin{document}
\makecvtitle

\section{Istruzione}
\cventry{2010--2015}{Diploma di liceo scientifico}{ISIS L.Calabrese P.Levi}{San Pietro in Cariano (VR)}{\textit{80/100}}{}
\cventry{2015--2018}{Laurea in Informatica}{Università degli studi di Verona}{Verona}{\textit{110/110 con Lode}}{}
\cventry{2018--oggi}{Laurea magistrale in Ingegneria e scienze informatiche}{Università degli studi di Verona}{Verona}{}{}

\section{Esperienze}
\cventry{2018--oggi}{Tutor per il corso ``Sfide di Programmazione''}{Università degli studi di Verona}{}{}{}
\cventry{2018--oggi}{Tutor per il corso ``Laboratorio Programmazione per Matematica''}{Università degli studi di Verona}{}{}{}
\cventry{2018--oggi}{Tutor agli allenamenti delle Olimpiadi Italiane di Informatica}{}{}{}{}
\cventry{2018}{Partecipazione alle qualificiazioni per le gare ICPC-ACM SWERC}{Télécom ParisTech}{Parigi}{}{}
\cventry{2019}{Partecipazione alle finali mondiali della gara Google HashCode}{Google European headquarter}{Dublino}{}{}

\section{Lingue}
\cvitemwithcomment{Inglese}{Livello B1}{}

\section{Competenze informatiche}
\cvitem{Programmazione}{C, C++, C\#, Java, Python, JavaScript, TypeScript}
\cvitem{Sistemi operativi}{Linux (Ubuntu, Debian, ArchLinux), Windows, MacOS, Android}
\cvitem{Altre competenze}{sviluppo web (HTML5, CSS, React), database relazionali (MySQL, PostgreSQL, SQLite), amministrazione di sistemi Linux/UNIX-like}
\cvitem{competitive programming}{Ho svolto a gare di programmaizone competitiva quali Google HashCode ed ICPC-ACM SWERC}

\end{document}
