\documentclass[11pt,a4paper,sans]{moderncv}
\usepackage[italian]{babel}
\usepackage[utf8]{inputenc}
\usepackage[margin=0.70in]{geometry}

\moderncvstyle{classic} 
\moderncvcolor{blue} 

\name{Alessandro}{Righi}
\title{Curriculum vitae}
\address{Via A.Volta 9}{San Pietro in Cariano (VR)}{37029}
\phone[mobile]{+39~349~843~7935}
\email{alessandro.righi@outlook.it}
\homepage{alerighi.github.io}
\social[linkedin]{alerighi}
\social[github]{alerighi}

\begin{document}
\makecvtitle

\section{Istruzione}
\cventry{2010--2015}{Diploma di liceo scientifico}{ISIS L.Calabrese P.Levi}{San Pietro in Cariano (VR)}{\textit{80/100}}{}
\cventry{2015--2018}{Laurea in Informatica}{Università degli studi di Verona}{}{\textit{110/110 con Lode}}{}
\cventry{2018--oggi}{Laurea magistrale in Ingegneria e scienze informatiche, indirizzo Sicurezza}{Università degli studi di Verona}{}{}{}

\section{Esperienze lavorative}
\cventry{2018--oggi}{Tutor e sviluppatore software presso il team Olimpiadi italiane di Informatica}{AICA}{}{}{}
\cventry{2018--oggi}{Tutoraggio per corsi di programmazione}{Università degli studi di Verona}{}{}{}


\section{Esperienze formative}
\cventry{2018}{Partecipazione alle qualificiazioni per le gare di competitive programming ICPC-ACM SWERC}{Télécom ParisTech}{Parigi}{}{}
\cventry{2019}{Partecipazione alle finali mondiali della gara di competitive programming Google HashCode}{Google European headquarter}{Dublino}{}{}
\cventry{2019}{Team leader del team italiano alle gare CEOI 2019}{}{Bratislava, Slovacchia}{}{}

\section{Lingue}
\cvitemwithcomment{Inglese}{Livello B1}{}

\section{Competenze informatiche}
\cvitem{Linguaggi di programmazione}{C, C++, Rust, Python, Java, Kotlin, C\#, JavaScript, TypeScript, PHP, Bash, assembly x86, programmazione funzionale}
\cvitem{Sistemi operativi}{Ottima conoscenza di GNU/Linux (Debian/Ubuntu, ArchLinux, RHEL/CentOS), buona conoscenza di Windows, MacOS ed Android}
\cvitem{Amministrazione di sistema}{Amministrazione di sistemi GNU/Linux, conoscenza di tools quali systemd, iptables, LVM, container Docker/LXC, virtualizzazione, NGINX, Apache, wireguard}
\cvitem{Sicurezza informatica}{Conoscenza dei protocolli più diffusi (SSL, HTTPS, SSH, PGP/GPG), conoscenza del funzionamento degli algoritmi crittografici (RSA, Diffie-Hellman, AES)}
\cvitem{Algoritmi}{Buona conoscenza dei principali algoritmi e capacità di svilupparne di nuovi, partecipazione a gare di programmazione competitiva}
\cvitem{Altre competenze}{Buona conoscenza dei database relazionali (PostgreSQL, MySQL, SQLite), sviluppo web (HTML5, CSS, framework quali Angular/React), sviluppo applicazioni Android}

\end{document}
